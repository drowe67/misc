\documentclass{article}
\usepackage{amsmath}
\usepackage{hyperref}
\usepackage{tikz}
\usepackage{float}

\begin{document}

\title{Analog Transmission of Vocoder Features}
\maketitle

Given a vector of vocoder features $\bf{f}$, use an autoencoder $E$ to map them to a dimension $d$ latent vector $\bf{z}$ where $d$ is even.  The magnitude of each element of $\bf{z}$ is constrained to a maximum of 1, but unlike digital modulation is continuusly valued and not constrained to a discrete set of points. For bandwidth efficient transmission over the channel the elements of $\bf{z}$ are mapped to $d/2$ complex symbols $\bf{q}$. Compared to classical digital modulation, the elements of $\bf{z}$ can be considered BPSK symbols (continuously valued, analog bits), and the elements of $\bf{q}$ analog QPSK symbols.

Our goal is to determine if reasonable speech quality can be obtained over a channel of bandwidth $B<3000$ Hz and SNR between 0 and 6dB, both comparable to Single Side Band (SSB) - a common power and bandwidth efficient form of analog radio communication.

\section{Noise Simulation}

The autoencoder output $\bf{z}$ is updated every $T_z=1/R_z$ seconds, giving a BPSK symbol rate of:
\begin{equation}
R_b=d/T_z
\end{equation}
For example with $T_z=0.04, d=80, R_b=2000$ symbols/s.  The QPSK symbol rate is given by:
\begin{equation}
R_q = \frac{d}{2T_z} 
\end{equation}
For example with $T_z=0.04, d=80, R_q=1000$ symbols/s

We wish to simulate a channel of known $E_b/N_0$, where $E_b$ is the energy of each BPSK symbol, and $N_0$ is the noise power per unit bandwidth.  The energy of each BPSK symbol $E_b$ is the symbol power $S$ divided by the symbol rate $R_b=1/T_b$.  The noise per unit bandwidth is the total noise power $N$ divided by the bandwidth $B$ of the system.  If we are simulating at one sample per symbol, $B=R_b$:
\begin{equation}
\begin{split}
\frac{E_b}{N_0} &= \frac{S/R_b}{N/R_b} \\
                &= \frac{S}{N} \\
                &= \frac{A^2}{\sigma^2}
\end{split}
\end{equation}
where $A$ is the amplitude of each BPSK symbol and $\sigma^2=N$ is the variance (mean noise energy per sample).  If the noise is zero mean, we can estimate $\sigma^2$ over $K$ noise sample $r_i$:
\begin{equation}
\sigma^2 = \frac{1}{K}\sum_{i=0}^{K-1}r_i^2
\end{equation}
We generate noise samples $r_i$ by sampling a unit variance, zero mean Gaussian noise source $\mathcal{N}_i(0,1)$ that is scaled by $\sigma$:
\begin{equation}
\begin{split}
r_i &= \sigma \mathcal{N}_i(0,1) \\
\sigma &= \frac{A}{\sqrt{Eb/N_0}}
\end{split}
\end{equation}

In practice we add noise to the QPSK symbols.  Due to orthogonality, this can be viewed as two independant channels that have the same noise power:
\begin{equation}
r_i = \sigma \mathcal{N}_{2i}(0,1) + j\sigma\mathcal{N}_{2i+1}(0,1)
\end{equation}

\section{SNR Measurement}

It is useful to formulate expressions for estimating SNR from the BPSK and QPSK symbols.  The Signal to Noise ratio (SNR) is given by:
\begin{equation}
\label{eq:snr_theory}
\begin{split}
\frac{S}{N} &= \frac{E_bR_b}{N_0B} \\
            &= \frac{E_qR_q}{N_0B}
\end{split}
\end{equation}

At one sample per symbol, the power, the mean energy of each QPSK symbol over a window of $K$ samples is given by:
\begin{equation}
E_q = Var(q_i) = \frac{1}{K}\sum_{i=0}^{K-1}|q_i|^2
\end{equation}
As each QPSK contains 2 BPSK symbols, then energy is split evenly:
\begin{equation}
E_b = E_q/2 = Var(q_i)/2
\end{equation}
For example if the symbol amplitude is $A=1, E_b=A^2=1$, then $E_q=1+1=2$.

For transmission over multipath channels we arrange the QPSK symbols as $N_c$ parallel carriers, each running at a symbol rate of $R_s$ symbols/s, where $R_s$ is chosen based on delay spread considerations and is typically around 50 Hz.  However the OFDM carriers are arranged such that the total QPSK symbol rate over the channel $R_q$ remains constant, so for a given carrier power $S$, $E_q=S/R_q$ remains constant.  The expressions above can be used as a check for the noise injection process:
\begin{enumerate}
\item Set a target $E_b/N_0$ for the simulation run.
\item Establish the target SNR from (\ref{eq:snr_theory}) evaluated using the target $E_b/N_0$.
\item Measure $Eq=Var(q_i)$ over a sample of transmitted symbols.
\item Calculate measured SNR using (\ref{eq:snr_theory}) and compare. 
\end{enumerate}

\section{Glossary}

\begin{table} [H]
\centering
\begin{tabular}{l l l}
 \hline
 Symbol & Explanation & Units \\
 \hline
 $B$ & noise or signal bandwidth & Hz \\
 $d$ & dimension of latent vector $bf{z}$ \\
 $E_b/N_0$ & energy per BPSK symbol on spectral noise density \\
 $E_q/N_0$ & energy per QPSK symbol on spectral noise density \\
 $N_c$ & Number of carriers  \\
 $\bf{q}$ & vector of QPSK symbols \\ 
 $q_i$ & single QPSK symbol, element of $\bf{q}$ \\ 
 $R_b$ & BPSK symbol rate & symbols/second \\
 $R_q$ & QPSK symbol rate & symbols/second \\
 $R_s$ & per carrier QPSK symbol rate & symbols/second \\
 $R_z$ & latent vector update rate & Hz \\
 $T_b$ & BPSK symbol period & seconds \\
 $T_q$ & QPSK symbol period & seconds \\
 $T_s$ & per carrier QPSK symbol period & seconds\\
 $T_z$ & time between latent vector updates & seconds\\
 $SNR$ & signal to noise Ratio \\
 $S$ & signal power & Watts \\
 $N$ & noise power & Watts \\
 $\bf{z}$ & Autoencoder output latent vector \\ 
 $z_i$ & single latent vector element of $\bf{z}$, a BPSK symbol \\ 
 \hline
\end{tabular}
\caption{Glossary of Symbols}
\end{table}

\end{document}
