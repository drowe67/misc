\documentclass{article}
\usepackage{amsmath}
\usepackage{hyperref}
\usepackage{tikz}
\begin{document}

\title{Electro Magnetic Interference and HF Radio Receivers}
\maketitle

\section{Introduction}
 
Throughout the 20th century HF radio receiver technology evolved with the aim of detecting distant, narrow band signals.  Radio noise sets the limits of detection \cite{itu372}, and was assumed to be from natural sources or the receiver itself. However in the early 21st century, from lower HF right up to UHF noise from human activity now dominates the detection problem.  We are no longer seeking to maximise signal-to-noise ratio (SNR), but signal to human noise ratio (SHR).

Our target radio signals are from distant sources, weak, narrow band, and evolve slowly in time. Human generated radio noise tends to be very strong, wideband from narrow pulse sources, with a time domain envelope that evolves quickly in time.  It is often radiated from nearby sources, such as electricity transmission lines and house wiring a few 10s of meters from our antenna.  It has significant structure, so can be interpreted as an interfering signal, rather than random noise.

Modern radios are very good at rejecting unwanted (but well behaved) signals, using frequency selectivity and good strong signal performance.  This report explores ways we can tune out human made noise in the lower HF bands.

\section{Pulse Width Modulation}

The PWM switch mode power supply is a ubiquitous source of interference.  Consider the Fourier Series of an ideal pulse train \cite{wikipedia_pulse}:
\begin{equation} \label{eq_pwm}
\begin{split}
x(t) &= Ad+\frac{2A}{\pi} \sum_{n=1}^{\infty} \frac{sin(\pi n d)}{n}cos(n \omega t) \\
     &= Ad+\frac{2A}{\pi} \sum_{n=1}^{\infty} \frac{x(t)}{n} cos(n \omega t)
\end{split}
\end{equation}
where $0<d<1$ is the duty cycle and $\omega=2 \pi f$ the fundamental frequency.  The Fourier series is a sequence of harmonics $cos(n \omega t)$, with the amplitude of each harmonic set by the $x(t)/n$ term. A typical value for $f$ is a few 10's of kHz to several hundred kHz. For example with f = 70kHz and $n=101$ we will have a harmonic at 7.1 MHz. The $1/n$ factor means the power of each harmonic falls off slowly with frequency, e.g. the $n=100$ harmonic will be just $10log_{10}(1/100)=-20$ dB down compared to the fundamental ($n=1$) power, and the $n=101$ harmonic almost the same power at $10log_{10}(1/101)=-20.043$ dB down.

If the duty cycle $d$ is constant, then each harmonic is an unmodulated sine wave of constant amplitude. In practice $d$ is time varying, as the duty cycle is continuously adjusted by the power supply.  This leads to modulation of each harmonic $x(t)$, spreading the power to frequencies either side of the harmonic centre $n \omega$.

Let $d$ have a constant and time varying component:
\begin{equation}
d=d_c+ad(t)
\end{equation}
where $d(t)$ is a normalised modulation function and $a$ is the peak amplitude of the modulation (ie the peak jitter of the PWM signal). For the $n$-the harmonic:
\begin{equation} \label{eq:pwm_n}
\begin{split}
x_n(t) &= sin(\pi n d) \\
       &= sin(\pi n (d_c+ad(t)) \\
       &= sin(\pi n d_c + \pi n a d(t)) \\
       &= sin(\pi n d_c + h d(t))  \\
     h &= \pi n a
\end{split}
\end{equation}
where h is the peak amplitude of the modulation. It can be seen that $h$ is strong function of $a$, as small changes are multiplied by $n \pi$. For example for $n=100, a=0.01, h = \pi$.  Thus with just 1\% jitter of the PWM signal, a cycle of $d(t)$ would sweep the entire $\pm \pi$ range of the $sin()$ function.

We would like to estimate the spectrum of $x(t)$. First we consider small $h << 1$:
\begin{equation} \label{eq:small_h}
\begin{split}
x_n(t) &= sin(\pi n d_c)cos(hd(t)) + cos(\pi n d_c)sin(hd(t)) \\
       &\approx sin(\pi n d_c)(1 - hd^2(t)/2) + cos(\pi n d_c)hd(t) \\
       &= sin(\pi n d_c) + hcos(\pi n d_c)d(t)
\end{split}
\end{equation}
The LHS is a constant term related to the mean duty cycle of the PWM signal, the RHS is modulation term. When multiplied by the $n$-th harmonic $cos(n \omega t)$ in (\ref{eq_pwm}):
\begin{equation} \label{eq:small_h}
\begin{split}
x_n(t)cos(n \omega t) &= \left[ sin(\pi n d_c) + hcos(\pi n d_c)d(t) \right] cos(n \omega t) \\
                      &= sin(\pi n d_c)cos(n \omega t)) + hcos(\pi n d_c)d(t)cos(n \omega t) \\
                      &= c_1 cos(n \omega t)) + c_2 d(t)cos(n \omega t) \\
                   c_1 &= sin(\pi n d_c) \\
                   c_2 &= hcos(\pi n d_c)
\end{split}
\end{equation}
By examining the different frequency terms we can estimate the spectrum of the PWM signal near the n-th harmonic. Note $c_1$ and $c_2$ are constants that do not affect the number of different frequency termss. The LHS of (\ref{eq:small_h}) is a constant carrier term at the harmonic centre. The product of $d(t)cos(n \omega t))$ will produce images of the spectra of d(t) either side of $n \omega$. This is Amplitude Modulation (AM).  For example if $d(t)$ is a random variable with maximum frequency $\omega_m$ radians, we would see a uniform noise spectrum over the interval $n \omega \pm \omega_m$, with a central "carrier" spectral line at $n \omega$. If $\omega_m > \omega/2$, the entire region between each harmonic will have additive white noise generated from the PWM signal.

For larger $h$, Equation \ref{eq:pwm_n} can be interpreted as Phase Modulation (PM) \cite{wikipedia_phase} of a zero frequency carrier. Assuming a random $d(t)$, the resulting spectra of $x(t)$ can be approximated using Carsons Rule, as having 98\% of it's power contained within $B_c=2(h+1)f_m$ Hz, where $f_m$ is the maximum frequency component of $d(t)$. When multiplied by the $cos(n \omega t)$ term, the PM signal will spread the power of the carrier over range of adjacent frequencies, replacing the PWM carrier in the small $h$ case with a band of white noise.

For example, let $a=0.02$ (a few percent jitter of the PWM duty cycle) and the bandwidth of the PWM control loop be $f_m=3$ kHz. At $n=100$ we have $h=\pi n a = 2\pi$, which implies a random deviation of $\pm 2\pi$ (twice the entire phase range of $sin()$), and an $x(t)$ bandwidth of $B_c=2(2 \pi+1)f_m = (4\pi+2)f_m = 43$ kHz.

Given $h$ increases with $n$, we can expect to see AM type modulation either side of central harmonics lines at lower frequencies.  At intermediate frequencies a raised noise floor will be observed between harmonics.  As the frequency increases further the power of the harmonics lines will reduce and gradually be replaced by broadband noise.
\bibliographystyle{plain}
\bibliography{emi_refs}
\end{document}
