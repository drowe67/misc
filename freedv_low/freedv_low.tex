\documentclass{article}
\usepackage{amsmath}
\usepackage{array}
\begin{document}

\title{Low SNR FreeDV Mode}
\author{David Rowe VK5DGR}
\maketitle

\section{Introduction}

After 10 years development and on air experience with various FreeDV waveforms, we would like to develop a new waveform that outperforms and replaces a variety of existing modes such as 700C/D/E and 1600.  Requirements include \cite{freedv-020}:
\begin{enumerate}
\item Better performance than SSB at 0dB SNR on MPP and MPD channels.
\item A single mode that can handle MPP and MPD, and replace several existing FreeDV modes, simplifying the end user experience.
\item For compliance with Export Control regulations, the minimum speech codec bit rate is 700 bit/s.
\end{enumerate}

\section{Theory}

In this section we will develop a theoretical model to help us explore performance limits.  For practical PTT voice systems algorithmic delay is limited to a few 100ms, which limits the FEC codeword size and hence the performance of the code.  For PSK channels a threshold $E_b/N_0=2dB$ and a code rate $R=0.5$ is typical, where $E_b/N_0$ is the energy per payload data bit (coded $E_b/N_0$) in dB.  The lowest (threshold) SNR possible is given by:
\begin{equation}
SNR = \frac{E_b}{N_0} + 10log_{10}\left(\frac{R_b}{B}\right)
\end{equation}
where $R_b$ is the payload data bit rate, and $B$ is the bandwidth in which we measure SNR.  Given $Rb=700$ and $B=3000$ we have:
\begin{equation}
\begin{split}
SNR &= 2 + 10log_{10}(700/3000) \\
    &= -4.3 dB
\end{split}
\end{equation}
This is ideal performance for an AWGN channel.  In practice we must allocate some power for synchronisation, such as pilot symbols used for frequency and phase estimation.  The phase, frequency, and timing estimators estimators are not perfect, which reduces performance.

Performance on a multipath channels is significantly worse, typically 5 dB.  In addition, we may allocate some carrier power to deal with intersymbol interference (for example a cyclic prefix in OFDM modems).

A more complete model is
\begin{equation}
SNR = \frac{E_b}{N_0} + 10log_{10}\left(\frac{R_b}{B}\right) + L_s + L_{il} + L_{cp}
\end{equation}
where $L_s$ is the loss from power allocated to synchronisation symbols, $L_{il}$ is the real world implementation loss, and $L_{cp}$ is the loss in SNR due to the power allocated to the cyclic prefix.

Consider a repeating sequence of PSK symbols, consisting of $N_p-1$ data symbols followed by one pilot symbol, giving $N_p$ symbols in total.  Each symbol has an energy of $E_s$ Joules.  Power is energy per unit time.  For a repeatng sequence of symbols symbols the total power is:
\begin{equation}
S = \frac{E_sN_p}{T_sN_p}
\end{equation}
where $T_s$ is the symbol period. The power allocated to the pilot symbols is:
\begin{equation}
S_p = \frac{E_s}{T_sN_p}
\end{equation}The loss in signal power due to pilot insertion is:
\begin{equation}
\begin{split}
L_p &= \frac{S - S_p}{S} \\
    &= \frac{E_sN_p}{T_sN_p} - \frac{Es}{T_sN_p}
\end{split}
\end{equation}

\section{Glossary}

\begin{table}[h]
\centering
\begin{tabular}{l m{8cm} }
 \hline
 Acronym & Explanation \\
 \hline
 AWGN & Additive White Gaussian Noise - a communications channel with flat frequency response and additive noise \\
 FEC & Forward Error Correction \\
 PTT & Push To Talk - voice communications where only one person is transmitting at any one time.  Common in two way radio but not mobile telephones  \\
 MPP & Multipath Poor channel, 1 Hz Doppler, 2ms delay spread, typical for US and Australian interstate propogation \\
 MPD & Multipath Disturbed channel, 2 Hz Doppler, 4ms delay spread, typical for UK Winter NVIS propogation \\
 \hline
\end{tabular}
\caption{Glossary of Acronyms}
\end{table}

\begin{table}[h]
\centering
\begin{tabular}{l l l}
 \hline
 Symbol & Explanation & Units \\
 \hline
 $B$ & Noise bandwidth & Hz \\
 $E_b/N_0$ &  Energy per bit on spectral noise density & dB \\
 $R_b$ & Bit rate & Bits/s \\
 $SNR$ & Signal to Noise Ratio & dB \\
 \hline
\end{tabular}
\caption{Glossary of Symbols}
\end{table}

TODO could have expression for Fading channels, block error rate

 and we may need an increased pilot symbol rate, and a 

TODO check performance of a few codes

Table explaining MPP MPD

TODO: set up model, then discuss various algorithms and how they affect model

Where we can gain, diversity, PAPR reduction, reduced overheads for fast fading and ISI (discuss)

Challenges, fast fading, ISI

Initial goal, single waveform that handles MPP and MPD at low SNR, better than 700D performance

Joint low latency/high latency on two carriers ()diveristy

Wades techniques (ref)

\bibliographystyle{plain}
\bibliography{freedv_low_refs}
\end{document}
