\documentclass{article}
\usepackage{amsmath}
\usepackage{hyperref}
\usepackage{tikz}
\usetikzlibrary{arrows}
\usepackage{float}

\begin{document}

\title{Analog Transmission of Vocoder Features}
\maketitle

Given a vector of vocoder features $\bf{f}$, use an autoencoder $E$ to map them to a dimension $d$ latent vector $\bf{z}$ where $d$ is even.  The magnitude of each element $z_i$ of $\bf{z}$ is constrained to a maximum of 1, but unlike digital modulation is continuously valued and not constrained to a discrete set of points. For bandwidth efficient transmission over the channel the elements of $\bf{z}$ are mapped to $d/2$ complex symbols $\bf{q}$. Compared to classical digital modulation, the elements of $\bf{z}$ can be considered BPSK symbols (continuously valued, analog bits), and the elements of $\bf{q}$ analog QPSK symbols.

Our goal is to determine if reasonable speech quality can be obtained over a channel of bandwidth $B<3000$ Hz and SNR (measured in $B=3000$ Hz) of between 0 and 6dB, roughly the lower limit of Single Side Band (SSB) - a common power and bandwidth efficient form of analog radio communication.

\section{Simulation of AWGN Channels}

The autoencoder output $\bf{z}$ is updated every $T_z=1/R_z$ seconds, giving a BPSK symbol rate of:
\begin{equation}
R_b=d/T_z
\end{equation}
For example with $T_z=0.04, d=80, R_b=2000$ symbols/s.  The QPSK symbol rate is given by:
\begin{equation}
R_q = \frac{d}{2T_z} 
\end{equation}
For example with $T_z=0.04, d=80, R_q=1000$ symbols/s.

\begin{figure}[H]
\caption{Real sampled off-air signal.  We are interested in the blue bandpass interval of bandwidth $B$, which is single sided and hence complex valued. After shifting to baseband, it's power is unchanged, and it remains complex valued.}
\vspace{5mm}
\label{fig:bandpass}
\begin{center}
\begin{tikzpicture}[>=triangle 45,x=1.0cm,y=1.0cm]
\draw[thick,->] (-6,0) -- (6,0) node [below, align=left, text width=3cm]{Frequency};
\draw[thick,->] (0,0) -- (0,3);
\draw (-3,0) -- (-3,2) -- (-1,2) -- (-1,0);
\draw[blue] (1,0) -- (1,2) -- (3,2) -- (3,0);
\draw[<->] (1,1) -- node [above]{$B$}(3,1); 
\draw[->] (2,-0.5)  node [below]{$\omega$} -- (2,0);
\end{tikzpicture}
\begin{tikzpicture}[>=triangle 45,x=1.0cm,y=1.0cm]
\draw[thick,->] (-6,0) -- (6,0) node [below, align=left, text width=3cm]{Frequency};
\draw[thick,->] (0,0) -- (0,3);
\draw (-5,0) -- (-5,2) -- (-3,2) -- (-3,0);
\draw[blue] (-1,0) -- (-1,2) -- (1,2) -- (1,0);
\draw[->] (0,-0.5)  node [below]{$\omega=0$} -- (0,0);
\end{tikzpicture}
\end{center}
\end{figure}

We wish to simulate an AWGN channel with a user-defined $E_b/N_0$, where $E_b$ is the energy of each BPSK symbol, and $N_0$ is the noise power per unit bandwidth.  Consider a real valued signal sampled off air (Figure \ref{fig:bandpass}).  We will follow convention and define signal and noise power in the ``single sided" bandpass interval of the frequency spectrum with bandwidth B centered on $\omega$.  As the interval is single sided, we must use complex valued quantities to represent it.

We wish to simulate a bandpass AWGN channel at baseband ($\omega=0$).  This implies a frequency shift of the complex valued signal, but the signal remains complex valued and it's power is unchanged. Note that even at baseband we must use complex valued quantities for the signal and noise.  We cannot represent a bandpass baseband signal with real valued quantities. The negative frequency component is redundant and after frequency shifting can be removed by filtering.  

The energy of each BPSK symbol $E_b$ is the signal power $S$ divided by the symbol rate $R_b=1/T_b$.  The noise per unit bandwidth is the total noise power $N$ divided by the bandwidth $B$ of the system.  If we are simulating at one sample per symbol, $B=R_b$:
\begin{equation}
\begin{split}
\frac{E_b}{N_0} &= \frac{S/R_b}{N/R_b} \\
                &= \frac{S}{N} \\
                &= \frac{A^2}{\sigma^2}
\end{split}
\end{equation}
where $A$ is the amplitude of each BPSK symbol and $\sigma^2=N$ is the variance of the complex valued noise (mean noise energy per sample).  Given a set point $E_b/N_0$:
\begin{equation}
\label{eq:noise_sigma}
\sigma = \frac{A}{\sqrt{E_b/N_0}}
\end{equation}
The complex noise sample $r_i$ can be generated as:
\begin{equation}
r_i = \frac{\sigma}{\sqrt{2}}(\mathcal{N}_{2i}(0,1) + j\mathcal{N}_{2i+1}(0,1))
\end{equation}
where $\mathcal{N}_i(0,1)$ is the $i-th$ sample of a unit variance, zero mean, real Gaussian noise source.  Note the noise power is split evenly between the real and imaginary arms. Our symbols passing through an AWGN channel can be simulated at complex baseband as:
\begin{equation}
\begin{split}
\hat{z}_i &= z_i + r_i \\
\hat{q}_i &= q_i + r_i
\end{split}
\end{equation}
If the noise is zero mean, we can estimate $\sigma^2$ over $K$ noise samples $r_i$ as:
\begin{equation}
\sigma^2 = E[|r_i|^2] = \frac{1}{K}\sum_{i=0}^{K-1}|r_i|^2 
\end{equation}

\section{SNR Measurement}

In order to compare with other methods of speech communication that have varying bandwidths $B$, it is useful to formulate expressions for estimating SNR from the BPSK and QPSK symbols.  The Signal to Noise ratio (SNR) is given by:
\begin{equation}
\label{eq:snr_theory}
\begin{split}
\frac{S}{N} &= \frac{E_bR_b}{N_0B} \\
            &= \frac{E_qR_q}{N_0B}
\end{split}
\end{equation}
A noise bandwidth $B$ needs to be selected; common choices are $B=R_b$, in which case $S/N=E_b/N_0$; for HF radio $B=3000$ Hz to compare with existing analog and digital voice waveforms; or $B=1$ to obtain a normalised $C/N_0$ carrier power to noise density ratio.
 
At one sample per symbol, the power, the mean energy of each QPSK symbol over a window of $K$ samples is given by:
\begin{equation}
E_q = E[|q_i|^2] = \frac{1}{K}\sum_{i=0}^{K-1}|q_i|^2
\end{equation}
As each QPSK symbol contains 2 BPSK symbols, the energy is split evenly:
\begin{equation}
E_b = E_q/2
\end{equation}
For example if the symbol amplitude is $A=1, E_b=A^2=1$, then $E_q=1+1=2$.

To model transmission over multipath channels using OFDM we arrange the QPSK symbols as $N_c$ parallel carriers, each running at a symbol rate of $R_s=R_q/N_c$ symbols/s, where $R_s$ is chosen based on delay spread considerations.  Typical values for HF modems are $N_c=20$ and $R_s=50$ Hz. However the OFDM carriers are arranged such that the total symbol rate over the channel remains constant.  So for a given signal power $E_q$ and $E_b$ remain constant (Table \ref{tab:constant_eb}).

\begin{table} [H]
\centering
\begin{tabular}{l l l l l l l}
 \hline
 Waveform            & $N_c$ & $R_s$ & $R_q$ & $R_b$ & $E_q$ & $E_b$ \\
 \hline
 Single Carrier BPSK & 1     & -  & -    & 2000  & -        & $S/2000$ \\
 Single Carrier QPSK & 1     & -  & 1000 & 2000  & $S/1000$ & $S/2000$ \\
 OFDM QPSK           & 20    & 50 & 1000 & 2000  & $S/1000$ & $S/2000$ \\
 \hline
\end{tabular}
\caption{$E_b$ and $E_q$ examples for single and multi-carrier OFDM waveforms for constant carrier power $S$}
\label{tab:constant_eb}
\end{table}

In order to evaluate the ML system early in the development process it is important to ensure the noise is correctly calibrated. The expressions above can be used to check the noise injection process:
\begin{enumerate}
\item Set a target $E_b/N_0$ for the simulation run, and calculate $\sigma$ using (\ref{eq:noise_sigma}).
\item Establish the equivalent target SNR from (\ref{eq:snr_theory}) evaluated using the target $E_b/N_0$.
\item After the simulation run measure $E_q=E[|q_i|^2]$ over a sample of transmitted symbols.  Note that in general $E_q \ne 2$ as the encoder outputs continuous values.
\item Calculate measured SNR using (\ref{eq:snr_theory}) and compare.
\end{enumerate}

The calibration of the noise injection can be checked by replacing the encoder output $z_i$ with discrete PSK symbols to create a digital modem, then measuring the BER at $E_b/N_0$ points, and comparing to the theoretical BER given by:
\begin{equation}
\begin{split}
BER_{awgn} &= 0.5erfc(\sqrt{E_b/N_0}) \\
BER_{multipath} &= 0.5 \left(1-\sqrt{\frac{E_b/N_0}{E_b/N_0+1}} \right)
\end{split}
\end{equation}
\section{Glossary}

\begin{table} [H]
\centering
\begin{tabular}{l l l}
 \hline
 Symbol & Explanation & Units \\
 \hline
 $B$ & noise or signal bandwidth & Hz \\
 $d$ & dimension of latent vector $\bf{z}$ \\
 $E_b/N_0$ & energy per BPSK symbol on spectral noise density \\
 $E_q/N_0$ & energy per QPSK symbol on spectral noise density \\
 $N$ & total noise power & Watts \\
 $N_c$ & Number of carriers  \\
 $\bf{q}$ & vector of QPSK symbols \\ 
 $q_i$ & single QPSK symbol, element of $\bf{q}$ \\ 
 $R_b$ & BPSK symbol rate & symbols/second \\
 $R_q$ & QPSK symbol rate & symbols/second \\
 $R_s$ & OFDM per carrier QPSK symbol rate & symbols/second \\
 $R_z$ & latent vector update rate & Hz \\
 $SNR$ & signal to noise Ratio \\
 $S$ & total signal (carrier) power & Watts \\
 $T_b$ & BPSK symbol period & seconds \\
 $T_q$ & QPSK symbol period & seconds \\
 $T_s$ & OFDM per carrier QPSK symbol period & seconds\\
 $T_z$ & time between latent vector updates & seconds\\
 $r_i$ & noise sample \\
 $\bf{z}$ & Autoencoder output latent vector \\ 
 $z_i$ & single latent vector element of $\bf{z}$, a BPSK symbol \\ 
 \hline
\end{tabular}
\caption{Glossary of Symbols}
\end{table}

\end{document}
