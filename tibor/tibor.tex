\documentclass{article}
\usepackage{amsmath}
\usepackage{hyperref}
\usepackage{tikz}
\begin{document}

\title{Tibor's Questions}
\maketitle

\section{Bandwidth of baseband BPSK}

Why is the bandwidth of baseband BPSK (ASK) signal $R_s/2$, but we need bandwidth $R_s$ for a passband BPSK signal?

From sampling theory any bandpass signal of bandwidth B can be exactly represented by a sequence of complex samples at sample rate B Hz, e.g. a QPSK signal of 1000 symbols/s at 1 MHz requires a sample rate of 1000 Hz (min) to represent it.

Consider the special case of a real-valued baseband BPSK sequence $x = e^{\pm j \pi}$ at sample rate $R_s$.  From sampling theory a real sequence is a special case of a complex sequence so it still occupies a bandwidth $R_s$, centred on 0 Hz, between $-R_s/2$ and $R_s/2$. For real valued sequences the Fourier Transform $X(\omega)$ is conjugate symmetric:
\begin{equation}
X(\omega) = X^*(-\omega)
\end{equation}
So for real valued baseband BPSK signals we only need bandwidth $R_s/2$ of the spectrum ($0..R_s/2$, or $-R_s/2..0$) to represent it.  Given either half, we can reconstruct it perfectly.

A property of a real signal is $Re\{x\} = x$, e.g. for the example above $Re\{x\} = x = \pm 1$. Therefore if this property holds, the signal is real, and we only require bandwidth $R_s/2$ to perfectly represent the signal.
            
However if we apply a non-zero frequency or phase offset, it breaks symmetry, e.g. consider a phase shift applied to $x$:
\begin{equation}
\begin{split}
y &= x e^{j \pi/2} \\
  &= e^{\pm j3 \pi /2} \\
  &= \pm j \\
Re\{y\} &= 0 \ne y
\end{split}
\end{equation}

So the signal $y$ can only be represented by a complex valued sequence, and its spectrum is no longer symmetrical.  As the sequence is still rate $R_s$, from sampling theory the signal must have a bandwidth $R_s$.

\section{Modulation Stripping of analog PSK}

Consider an analog PSK symbol $x$ that is modulated purely along the real axis, and has an unknown phase offset $\phi$:
\begin{equation}
x = ae^{j\phi}
\end{equation}
where $-1<a<1$ is continuously valued.  In digital modulation, a technique called modulation stripping is used to resolve the phase offset. Typically the received BPSK symbols are squared, and QPSK symbols taken to the fourth power, which removes this effect of the constellation phase, leaving just the phase offset.  Applied to our real-modulated analog PSK symbol:
\begin{equation}
\begin{split}
x^2 &= |a|^2e^{j2\phi} \\
arg[x^2] &= 2\phi
\end{split}
\end{equation}
which allows us to resolve the unknown phase offset $\phi$.  Not that unlike digital modulation, the symbol $x$ can take any position on the real axis on the interval $-1<a<1$.

Now consider a generalised quadrature symbol $y=a+jb=me^{j\theta}$ comprised of two orthogonal, continuously valued real symbols $-1<a<1$ and $-1<b<1$.  Using the analog to QPSK, lets try applying the fourth power:
\begin{equation}
\begin{split}
y^4 &= |m|^4e^{j4\theta+4\phi} \\
arg[y^4] &= 4\theta+4\phi
\end{split}
\end{equation}
Which is unsuccessful in resolving the phase offset, as the phase of the complex symbol $\theta$ is unknown.
\end{document}
